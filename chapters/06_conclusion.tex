\chapter{Conclusion}
In this thesis, a new pipeline for TILs scoring in breast cancer patients was presented.
The tasks of tissue segmentation and TILs detection for WSI were approached with DeepLabv3+
which is actively used in computational pathology, as well as with a novel transformer-based
approach, SegFormer. As a result, the SegFormer was found to be superior in the TILs
segmentation task. The analysis of TCGA-BRCA slides with the resulting models, allowed
for experiments with different TILs density based features. Several of them showed the
ability to successfully stratify patients at multiple prevalences. This proves a strong
signal of TILs in the TCGA-BRCA dataset, which was not previously demonstrated while
analyzing exclusively this publically available dataset. 

As an outlook, one of the main missing prospects of this work is the absence of
benchmarking with the existing challenge entries. Due to the late start of this
work the submission portal was closed by the time the models were present. 
The TiGER challenge team that achieved the highest rank on the preliminary leaderboards
for both segmentation of tissue and TILs published their approach before the challenge ended.
The authors~\cite{shephard2022tiager} trained Efficient-UNet-B0 for both tasks using Jaccard loss.
The model was trained via a stratified 5-fold
cross-validation framework with 512$\times$512 patches at 10$\times$ magnification,
with a stride of 256 pixels. Whereas for TILs segmentation,
the patches of 128$\times$128 were extracted at 20$\times$ magnification, with a stride of 100 pixels.
The selected model with the best F1-score across each experiment on 5-fold cross-validation achieved
a dice score of 0.762 for tumor segmentation, 0.718 for stroma segmentation, and  0.702 for TILs
detection. The TILs score was calculated by dividing the total predicted TILs area by the stromal
area and multiplying by 100. The score was further constrained to be an integer between 0 and 100.
The resulting TILs scores got the highest concordance of 0.719 for predicting survival as part of
a Cox proportional hazards model. Right now any of the results in this paper or any future papers
of the TiGER challenge can be compared to the results in this thesis. Due to the arrangement of the
challenge, the reported results are based on hidden test set and internal clincal data.
As long as the submission server is closed, those models and ideas need to be reimplemented.
Besides benchmarking, those challenge entries can provide valuable ideas for improvements to the current pipeline.
Such as applying transfer-learning, by using this model with pre-trained weights from ImageNet for tissue
segmentation or an on-the-fly under-sampling approach to alleviate class imbalance of TILs segmentation task.
There are also some parallels in approaches such as, similarly to this thesis, the authors~\cite{shephard2022tiager}
viewed TILs detection problem as TILs segmentation, but additionally, applied stain augmentation. Even though the
importance of the stain addition could be
suspected from the tissue segmentation results, this is a valuable experiment that should be performed.
All in all, this thesis also provides a better understanding and test possibilities for TILs features development.
Instead of blindly submitting a value, adapting the TCGA-BRCA data set enables it to actually view the data specifics
which could have helped the participants of the challenge (applicable if of course TCGA-BRCA data is not included in
the hidden clinical data).