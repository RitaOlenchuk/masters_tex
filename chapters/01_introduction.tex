% !TeX root = ../main.tex
% Add the above to each chapter to make compiling the PDF easier in some editors.

\chapter{Introduction}\label{chapter:introduction}
Cancer causes changes in tissue at the sub-cellular scale. Pathologists examine a tissue specimen
under a powerful microscope to look for abnormalities which indicate cancer. This manual
process has traditionally been the de facto standard for diagnosis and grading of cancer tumors.
While it continues to be widely applied in clinical settings, manual examination of tissue is a
subjective, qualitative analysis and is not scalable to translational and clinical research studies
involving hundreds or thousands of tissue specimens. A quantitative analysis of normal and tumor
tissue, on the other hand, can provide novel insights into observed and latent sub-cellular tissue characteristics and can lead to a better understanding of
mechanisms underlying cancer onset and progression
(Madabhushi, 2009; Chennubhotla et al., 2017;
Cooper et al., 2018). 
%Methods for Segmentation and
Classification of Digital Microscopy
Tissue Images
Quoc Dang Vu1
