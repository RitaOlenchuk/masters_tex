% !TeX root = ../main.tex
% Add the above to each chapter to make compiling the PDF easier in some editors.

\chapter{Introduction}\label{chapter:introduction}
Cancer causes changes in tissue at the sub-cellular scale. Pathologists examine a tissue specimen
under a powerful microscope to look for abnormalities which indicate cancer. This manual
process has traditionally been the de facto standard for diagnosis and grading of cancer tumors.
While it continues to be widely applied in clinical settings, manual examination of tissue is a
subjective, qualitative analysis and is not scalable to translational and clinical research studies
involving hundreds or thousands of tissue specimens. A quantitative analysis of normal and tumor
tissue, on the other hand, can provide novel insights into observed and latent sub-cellular tissue characteristics and can lead to a better understanding of
mechanisms underlying cancer onset and progression
(Madabhushi, 2009; Chennubhotla et al., 2017;
Cooper et al., 2018). 
%Methods for Segmentation and
Classification of Digital Microscopy
Tissue Images
Quoc Dang Vu1
https://www.sciencedirect.com/science/article/pii/S0923753419313699

Breast cancer is one of the most common malignant tumours in women [1] and is still the second leading cause of cancer-related death in women around the world [2]
1 Siegel R, Naishadham D, Jemal A. Cancer statistics, 2013. CA Cancer J Clin. 2013;63(1):11–30..
2 Jemal A, Bray F, Center MM, Ferlay J, Ward E, Forman D. Global cancer statistics. CA Cancer J Clin. 2011;61(2):69–90.

Increasing evidence indicates that the tumour microenvironment plays an important role in tumour formation, growth, invasion and metastasis. Tumour-infiltrating lymphocytes (TILs) have emerged as potentially important prognostic and/or predictive biomarkers for breast cancer [4, 5].

4 Denkert C, Loibl S, Noske A, Roller M, Muller BM, Komor M, Budczies J, Darb-Esfahani S, Kronenwett R, Hanusch C, et al. Tumor-associated lymphocytes as an independent predictor of response to neoadjuvant chemotherapy in breast cancer. J Clin Oncol. 2010;28(1):105–13.
5 Loi S, Sirtaine N, Piette F, Salgado R, Viale G, Van Eenoo F, Rouas G, Francis P, Crown JP, Hitre E, et al. Prognostic and predictive value of tumor-infiltrating lymphocytes in a phase III randomized adjuvant breast cancer trial in node-positive breast cancer comparing the addition of docetaxel to doxorubicin with doxorubicin-based chemotherapy: BIG 02-98. J Clin Oncol. 2013;31(7):860–7.