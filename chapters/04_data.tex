\chapter{Data}
\section{Segmentation}
The data comes from publicly released Tumor InfiltratinG lymphocytes in
breast cancER (TiGER) challenge dataset containing
digital pathology images of Her2 positive (Her2+) and Triple Negative (TNBC) breast
cancer whole-slide images (WSI), regions of interest (ROIs) and manual annotations. More specifically, WSIROIS dataset
was used for model training, validation, and testing (see Table~\ref*{tab:segm_data}).
TiGER data, both at WSI and ROI level, was released at a spacing (pixel size) of approximately 0.5 \textmu m/px,
for more information please refer to the original challenge
website\footnote{\url{https://tiger.grand-challenge.org/Data/}}.

\begin{table}[h!]
\centering
\begin{tabular}{ l c c c c c c } 
\hline
\multirow{3}{*}{Source} &  \multicolumn{3}{c}{Tissue} & \multicolumn{3}{c}{TILs}\\ 
\cline{2-7}
 & \#slides & \#ROIs & median ROI size & \#slides & \#ROIs & median ROI size \\ 
  & & & \#pixels [k] & & & \#pixels [k] \\ 
\hline
TCGA-BRCA & 151 & 151 & 4 983 & 124 & 1744 & 20\\ 
RUMC & 26 & 81 & 1 312 & 26 & 81 & 1 312\\ 
JB & 18 & 54 & 1 465 & 18 & 54 & 1 465\\
\hline
 & 195 & 286 & & 168 & 1879 &\\
\end{tabular}
\caption{\label{tab:segm_data}TiGER data overview. Sources: Cancer Genome Atlas Breast Invasive Carcinoma (TCGA-BRCA),
Radboud University Medical Center (RUMC) and Jules Bordet Institute (JB). Tissue slides and ROIs refer to the segmentation
images and annotations whereas TILs prefix specifies the data for TILs detection provided by the challenge. }
\end{table}

The TiGER tissue annotations include eight
labels that were reduced to three (see Table~\ref*{tab:label_data}).
The training masks were generated using available XML-files. In the provided mask
images, in certain cases, regions not included in ROIs and non-annotated regions in ROIs where
marked with the same label, which could not be directly used for training.
\begin{table}[h!]
\centering
\begin{tabular}{ l c c c } 
\hline
TiGER Label & Share & ID & new ID  \\ 
\hline
Invasive tumor & 0.283 & 1 & 1 \\ 
In-situ tumor & 0.029 & 3 & 1 \\ 
Tumor-associated stroma & 0.286 & 2 & 2\\
Inflamed stroma & 0.096 & 6 & 2\\
Necrosis not in-situ & 0.048 & 5 & 0\\
Healthy glands & 0.0008 & 4 & 0 \\ 
Background & 0.231 & 0 & 0 \\ 
Rest & 0.026 & 7 & 0\\
\hline
\end{tabular}
\caption{\label{tab:label_data} Reduction of labels provided in TiGER challenge dataset. Resulting labels include three classes: Tumor (1), Stroma (2) and Rest (0) with shares of 0.312, 0.382 and 0.306. Shares were calculated by dividing
the number of pixels belonging to some label by the number of the pixel in the current image and averaged over all images. }
\end{table}

While for tissue segmentation the images and their masks could be used as provided in the 
dataset, the data for TILs segmentation required some preprocessing. The TiGER fixed-size bounding box
annotation for lymphocytes and plasma cells (see Table~\ref*{tab:tils_data}) was adapted for segmentation by transforming each
bounding box into an annotation of the center pixel with a dilatation of three.
\begin{table}[h!]
\centering
\begin{tabular}{ l c c c c c c } 
\hline
\multirow{2}{*}{Source} & & & & \multicolumn{3}{c}{Number of cells per ROI}\\ 
\cline{2-7}
 & \#slides & \#ROIs & \#cells & min & max & median \\ 
\hline
TCGA-BRCA & 124 & 1 744 & 19 115 & 0 (44.3\%) & 206 & 1\\ 
RUMC & 26 & 81 & 4 728 & 0 (7.4\%) & 657 & 19\\ 
JB & 18 & 54 & 5 523 & 0 (7.4\%) & 608 & 51.5\\
\hline
 & 168 & 1 879 & 29 366 & & &\\
\end{tabular}
\caption{\label{tab:tils_data} Data overview for TILs detection. Sources: Cancer Genome Atlas Breast Invasive Carcinoma (TCGA-BRCA),
Radboud University Medical Center (RUMC) and Jules Bordet Institute (JB). Number of cells here refers to the number of
bounding boxes that were assigned for lymphocytes and plasma cells, further named TILs.}
\end{table}

\section{Survival Analysis}
TiGER challenge aims to assess the prognostic significance of computer-generated TILs scores
for predicting survival applying Cox proportional hazards model. The survival analysis
is done internally, hence no corresponding data was released. The survival analysis
within this thesis is done exclusively on publicly available TCGA-BRCA data. Where death
(\texttt{vital\_status} = 1) is considered as an event, and the time until the event or censoring is
taken either from \texttt{days\_to\_death} (number of days to death from first diagnosis) or \texttt{days\_to\_followup}
(number of days to last follow-up from first diagnosis). 
\begin{table}[h!]
\centering
\begin{tabular}{ l c c c } 
\hline
\texttt{vital\_status} & \#cases & median age at diagnosis [years] & median time to event [months]\\ 
\hline
Dead & 146 & 62 & 37.8\\ 
Alive & 919 & 58 & 26.3\\ 
\hline
 & 1065 & 58 & 28.7\\
\end{tabular}
\caption{\label{tab:surv_data}Survival data overview.}
\end{table}
