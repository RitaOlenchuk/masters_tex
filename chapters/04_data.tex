\chapter{Data}
\section{Segmentation}
The data comes from publicly released Tumor InfiltratinG lymphocytes in
breast cancER (TiGER) challenge dataset containing
digital pathology images of Her2 positive (Her2+) and Triple Negative (TNBC) breast
cancer whole-slide images (WSI), regions of interest (ROIs) and manual annotations. More specifically, WSIROIS dataset
was used for model training, validation and testing. This dataset includes images from three different sources:
Cancer Genome Atlas Breast Invasive Carcinoma (TCGA-BRCA),
Radboud University Medical Center (RUMC) and Jules Bordet Institute (JB) (Table~\ref*{tab:segm_data}).
TiGER data, both at WSI and ROI level, was released at a spacing (pixel size) of approximately 0.5 um/px,
for more information please refer to the original challenge
website\footnote{\url{https://tiger.grand-challenge.org/Data/}}.
\begin{table}[h!]
\centering
\begin{tabular}{ l c c c c } 
\hline
Source & \#tissue slides & \#tissue ROIs & \#TILs slides & \#TILs ROIs \\ 
\hline
TCGA-BRCA & 151 & 151 & 124 & 1744 \\ 
RUMC & 26 & 81 & 26 & 81 \\ 
JB & 18 & 54 & 18 & 54\\
\hline
 & 195 & 286 & 168 & 1879\\
\end{tabular}
\caption{\label{tab:segm_data}TiGER data overview.}
\end{table}
The training masks were generated by using provided XML-files. In the provided mask
images, in certain cases, regions not included in ROIs and non annotated regions in ROIs where
marked with the same class id, which could not be directly used for training.

While for tissue segmentation the images and their masks could be used as provided in the 
dataset, the data for TILs segmentation required some preprocessing. The TiGER fixed-size bounding box
annotation for lymphocytes and plasma cells was adapted for segmentation by transforming each
bounding box into an annotation of the center pixel with a dilatation of three.

\section{Survival Analysis}
TiGER challenge aims to assess the prognostic significance of computer-generated TILs scores
for predicting survival applying Cox proportional hazards model. The survival analysis
is done internally, hence no corresponding data was released. The survival analysis
within this thesis is done exclusively on publicly available TCGA-BRCA data. Where death
(\texttt{vital\_status} = 1) is considered as an event, and the time until the event or censoring is
taken either from \texttt{days\_to\_death} (number of days to death from first diagnosis) or \texttt{days\_to\_followup}
(number of days to last follow-up from first diagnosis). 
\begin{table}[h!]
\centering
\begin{tabular}{ l c c c } 
\hline
\texttt{vital\_status} & \#cases & median age at diagnosis [years] & median time to event [months]\\ 
\hline
Dead & 146 & 62 & 37.8\\ 
Alive & 919 & 58 & 26.3\\ 
\hline
 & 1065 & 58 & 28.7\\
\end{tabular}
\caption{\label{tab:surv_data}Survival data overview.}
\end{table}
