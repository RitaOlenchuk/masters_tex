\chapter{\abstractname}

Breast cancer is the most common type of cancer worldwide with a high mortality
rate causing millions of cancer-related deaths annually.
Optimization of diagnostic biomarkers play important role in the improvement
of breast cancer prognosis and its therapeutic outcomes.
The quantification of mononuclear immune cells that infiltrate tumor tissue,
named tumor-infiltrating lymphocytes (TILs) has been shown to
be a clinically useful biomarker and essential in breast cancer progression.
TILs analysis is typically performed by pathologists that estimate the proportion
of TILs in critical areas.

This thesis aimed to show that TILs evaluation can be automated with deep learning
based techniques and therefore developed a pipeline that segments tissue relevant
for TILs score, such as stromal and tumorous regions, and detects TILs. The tissue
segmentation model scored 0.85 dice score on the test set and TILs
detection was approached as segmentation with a novel for computational pathology
transformer based architecture - SegFormer, that scored 0.66 F1-score on the test set.
The resulting methods applied to TCGA-BRCA clinical data revealed the ability of TILs
densities in various stroma related regions to successfully stratify the patients and
as part of a Cox proportional hazards model achieved concordance of 0.59, p-value 0.00019
and 0.78 hazard ratio, that supports the assumption that the high level of TILs plays a
role in prolonged survival probability and it can be used as a predictor for prognosis
of breast cancer patients.



\makeatletter
\ifthenelse{\pdf@strcmp{\languagename}{english}=0}
{\renewcommand{\abstractname}{Kurzfassung}}
{\renewcommand{\abstractname}{Abstract}}
\makeatother

\chapter{\abstractname}

%TODO: Abstract in other language
\begin{otherlanguage}{ngerman} 
Brustkrebs ist weltweit häufigste Krebsart mit einer hohen Sterblichkeitsrate,
die jährlich Millionen von krebsbedingten Todesfällen verursacht.
Die Optimierung diagnostischer Biomarker spielt eine wichtige Rolle bei der
Verbesserung der Brustkrebsprognose und der therapeutischen Ergebnisse.
Die Quantifizierung mononukleärer Immunzellen, die das Tumorgewebe infiltrieren
und als tumorinfiltrierende Lymphozyten (TILs) bezeichnet werden, hat sich als
klinisch nützlicher Biomarker erwiesen, der für das Fortschreiten von Brustkrebs entscheidend ist.
Die Analyse der TILs wird in der Regel von Pathologen durchgeführt, die den Anteil der
TILs in kritischen Bereichen schätzen.

In dieser Arbeit sollte gezeigt werden, dass die Bewertung von TILs mit Hilfe von
Deep Learning Techniken automatisiert werden kann. Dazu wurde eine Pipeline entwickelt,
die Gewebe, die für die TILs-Bewertung relevant sind, wie stromale und tumoröse Regionen,
segmentiert und TILs erkennt. Das Gewebesegmentierungsmodell erzielte eine
Dice Wertung von 0.85 und die TILs-Erkennung wurde als Segmentierung mit einer für die
computergestützte Pathologie neuartigen, Trasnformer Architektur - SegFormer - durchgeführt,
die eine F1-Punktzahl von 0,66 auf dem Testset erzielte. Die resultierenden Methoden,
die auf die klinischen TCGA-BRCA-Daten angewandt wurden, zeigten die Fähigkeit der
TILs-Dichte in verschiedenen stromabezogenen Regionen, die Patienten erfolgreich zu
stratifizieren, und erreichten als Teil eines Cox Proportional Hazards Modells eine
Konkordanz von 0.59, einen p-Wert von 0.00019 und eine Hazard Ratio von 0.78, was die
Annahme stützt, dass das hohe Niveau von TILs eine Rolle bei der verlängerten
Überlebenswahrscheinlichkeit spielt und als Prädiktor für die Prognose von
Brustkrebspatientinnen verwendet werden kann.

\end{otherlanguage}


% Undo the name switch
\makeatletter
\ifthenelse{\pdf@strcmp{\languagename}{english}=0}
{\renewcommand{\abstractname}{Abstract}}
{\renewcommand{\abstractname}{Kurzfassung}}
\makeatother