\chapter{\abstractname}

Breast cancer is the most common type of cancer worldwide with a high mortality rate causing millions of cancer-related deaths annually.
Optimization of diagnostic biomarkers play important role in the improvement of breast cancer prognosis and its therapeutic outcomes.
At present, in early-stage disease clinicopathological risk stratification
is performed using a limited set of features such as tumor size and lymph node status, that do not stratify patients with sufficient granularity to permit selection for clinical trials. 
The results of high-throughput technology analysis have identified transcriptomic or proteomic features which are associated with particular histological features. 
Hence, the histological appearance of a tumor represents a useful cancer phenotype that can be further explored, and contribute to staging and stratification.

The quantification of mononuclear immune cells that infiltrate tumor tissue, named tumor-infiltrating lymphocytes (TILs) is a clinically useful biomarker and essential in breast cancer progression. TILs in tumors and the surrounding microenvironment are thought to reflect the ongoing anti-tumor immune response of the host.
TILs analysis is typically performed by pathologists that estimate the proportion of TILs in critical areas.

The automatic assessment of TILs by computational image analysis is useful for standardization and potential use in a clinical setting.
This thesis shows that TILs evaluation can be automated with deep learning based techniques and therefore developed a pipeline that segments tissue relevant for TILs score, such as stromal and tumorous regions, and detects TILs. The tissue segmentation model scored 0.85 dice score on the test set and TILs detection was approached as segmentation with a novel for computational pathology transformer based architecture - SegFormer, that scored 0.66 F1-score on the test set.

Application of machine learning approaches permitted the discovery of image based features that would be demanding to identify per hand, particularly if they only exist in small groups of patients. It made it feasible not only to show the effect of TILs density in stroma but experiment with various borders of stroma, tumor associated stroma, tumor itself and define novel heterogeneity features.
Applied to TCGA-BRCA clinical data, TILs related features demonstrated their ability to successfully stratify the patients and as part of a Cox proportional hazards model achieved concordance of 0.59, p-value 0.00019 and 0.78 hazard ratio, which supports the assumption that the high level of TILs plays a role in prolonged survival probability and it can be used as a predictor for prognosis of breast cancer patients.



\makeatletter
\ifthenelse{\pdf@strcmp{\languagename}{english}=0}
{\renewcommand{\abstractname}{Kurzfassung}}
{\renewcommand{\abstractname}{Abstract}}
\makeatother

\chapter{\abstractname}

%TODO: Abstract in other language
\begin{otherlanguage}{ngerman} 
Brustkrebs ist weltweit die häufigste Krebsart mit einer hohen Sterblichkeitsrate,
die jährlich Millionen von krebsbedingten Todesfällen verursacht.
Die Optimierung diagnostischer Biomarker spielt eine wichtige Rolle bei der Verbesserung der
Brustkrebsprognose und der therapeutischen Ergebnisse.
Gegenwärtig wird die klinisch-pathologische Risikostratifizierung in frühen Krankheitsstadien
mit einer begrenzten Anzahl von Merkmalen wie Tumorgröße und Lymphknotenstatus durchgeführt,
die die Patientinnen nicht mit ausreichender Granularität stratifizieren, um eine Auswahl für
klinische Studien zu ermöglichen. 
Die Ergebnisse der technologischen Hochdurchsatzanalyse haben transkriptomische oder
proteomische Merkmale identifiziert, die mit bestimmten histologischen Merkmalen verbunden sind. 
Daher stellt das histologische Erscheinungsbild eines Tumors einen nützlichen Krebsphänotyp
dar, der weiter erforscht werden und zur Stadieneinteilung und Stratifizierung beitragen kann.

Die Quantifizierung mononukleärer Immunzellen, die in das Tumorgewebe eindringen und als
tumorinfiltrierende Lymphozyten (TILs) bezeichnet werden, ist ein klinisch nützlicher Biomarker
und für das Fortschreiten von Brustkrebs von entscheidender Bedeutung. Es wird angenommen,
dass TILs in Tumoren und der umgebenden Mikroumgebung die laufende Anti-Tumor-Immunreaktion
des Wirts widerspiegeln.
Die Analyse der TILs wird in der Regel von Pathologen durchgeführt,
die den Anteil der TILs in kritischen Bereichen schätzen.

Die automatische Bewertung von TILs durch computergestützte Bildanalyse ist nützlich
für die Standardisierung und den potenziellen Einsatz im klinischen Umfeld.
Diese Arbeit zeigt, dass die Bewertung von TILs mit Deep-Learning-Techniken automatisiert
werden kann, und entwickelte daher eine Pipeline, die das für die TILs-Bewertung relevante Gewebe,
wie stromale und tumoröse Regionen, segmentiert und TILs erkennt. Das Gewebesegmentierungsmodell erzielte eine
Dice-Wert von 0.85 und die TILs-Erkennung wurde als Segmentierung mit einer für die
computergestützte Pathologie neuartigen, Trasnformer Architektur - SegFormer - durchgeführt,
die eine F1-Wert von 0.66 auf dem Testset erzielte.

Die Anwendung von Ansätzen des maschinellen Lernens ermöglichte die Entdeckung von
bildbasierten Merkmalen, deren Identifizierung von Hand sehr schwierig wäre,
insbesondere wenn sie nur bei kleinen Patientengruppen vorkommen. So konnte nicht
nur die Auswirkung der TILs-Dichte im Stroma aufgezeigt werden, sondern es konnte auch
mit verschiedenen Grenzen des Stromas, des tumorassoziierten Stromas und des Tumors
selbst experimentiert und neue Heterogenitätsmerkmale definiert werden.
Bei der Anwendung auf die klinischen TCGA-BRCA-Daten zeigten die TILs-bezogenen Merkmale
ihre Fähigkeit, die Patienten erfolgreich zu stratifizieren, und erreichten als Teil
eines Cox-Proportional-Hazards-Modells eine Konkordanz von 0.59, einen p-Wert von 0.00019 und
eine Hazard Ratio von 0.78, was die Annahme stützt, dass ein hohes Maß an TILs eine Rolle bei der
verlängerten Überlebenswahrscheinlichkeit spielt und als Prädiktor für die
Prognose von Brustkrebspatientinnen verwendet werden kann.

\end{otherlanguage}


% Undo the name switch
\makeatletter
\ifthenelse{\pdf@strcmp{\languagename}{english}=0}
{\renewcommand{\abstractname}{Abstract}}
{\renewcommand{\abstractname}{Kurzfassung}}
\makeatother